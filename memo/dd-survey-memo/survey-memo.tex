\documentclass[11pt,a4paper]{article}
\usepackage{microtype}\usepackage{mathptmx}
\usepackage{sdp_doc} % SDP style file
\usepackage[english]{babel}
\usepackage{listings}
\usepackage[pdfborderstyle={/S/U/W 1}]{hyperref}
\usepackage{graphicx}
\usepackage{amssymb,amsmath}
\usepackage{subfig}

\usepackage{xcolor}
\usepackage[mark]{gitinfo2}


% Configuration info
\newcommand{\bigdoctitle}{Is any existing data-driven programming model a viable solution for SDP?: a survey on available data-driven systems}
\newcommand{\docnr}{}
\newcommand{\context}{}
\newcommand{\revision}{git-\gitAbbrevHash}
\newcommand{\docauthor}{Montse Farreras, Daniele Lezi, Gladys Utrera, Jordi Fornes}
\newcommand{\leadauthor}{Montse Farreras}
\newcommand{\release}{Not released}
\newcommand{\docudate}{\gitAuthorDate}
\newcommand{\classification}{Unrestricted}
\newcommand{\docstatus}{Draft\xspace}
% Table with signatures
\newcommand{\signaturetable}{
  \begin{tabularx}{\textwidth}{|X|X|X|}
      \hline
      Name & Designation & Affilitation\\
      \hline
      Montse Farreras& Lead Author& University of Cambridge \\
      \hline
      Signature \& Date: & & \\
      & & \\
      & & \\
      \hline
      Name & Designation & Affilitation\\
      \hline
      & & \\
      \hline
      Signature \& Date: & & \\
      & & \\
      & & \\
      \hline
  \end{tabularx}
}

  % Table with version numbers
  \newcommand{\versiontable}{
  \begin{tabularx}{\textwidth}{|X|X|X|X|}
        \hline
        \bf{Version} & {\bf Date of issue} & {\bf Prepared by} & {\bf Comments}\\
        \hline
        0.0 & & & \\
        \hline
      \end{tabularx}
  }


\newcommand{\organisationtable}{
\begin{center}
 \sffamily{\bf ORGANISATION DETAILS}\end{center}
    \begin{table}[htbp]
      \centering
      \begin{tabular}[htbp]{|l|l|}
        \hline
        Name & Science Data Processor Consortium\\
        \hline
      \end{tabular}
    \end{table}
}

\newcommand{\Nfcorr}{\ensuremath{N_{\rm f,corr}}}


\title{}

\begin{document}
\sdpfrontpage

\section{Introduction}

The current architecture of the SKA SDP element~\cite{SDParch} envisages a 
data-driven model for the advandadges considered in~\cite{DDchoice}

The focus of this paper is to outline the SDP Execution Engine requirements
and survey available systems in accordance to the these 
requirements.

\subsection{The SDP challenge}
%Challenges:
The SDP software faces a number of challenges: it needs to achieve high-performance on key scientific algorithms in multi-PFLOPS regime so HPC technologies are critical. At the same time it needs to collect, manage, store and deliver vast amounts of data into viable products like Big Data processing so we care about variety, velocity, volume, veracity and Value of data.  It needs to Combine real-time and iterative execution environment and provide feedback at various cadence to other elements of the telescope (High Performance Data Analytics). The software needs to Operate 365 days a year, it needs to provide high availability and therefore to accommodate failure via software. Modern hyperscale environments
It needs to be Extensible, Scalable to Provide a modern eco-system to accommodate new algorithm development and upgrades
Extensibility, scalability, maintainability
SKA1 is the first ?milestone? ? expecting significant expansion in the 2020s. The lifetime of an observatory is around 50 years. 

\subsection{The SDP Software architecture}
% About the SDP software architecture
The SDP software architecture presented in~\cite{SDParch} has adopted a data-driven architecture. The key aspects of this architecture 
are: (i) the SDP processing is to be divided into a set of pure tasks, where each task will specify its input and output data. 
Tasks will be pure in the sense that no other data than the specified parameters will be used within a task. (ii) The SDP processing 
will be run in parallel taking into account the data dependencies between tasks. (iii) the assignment of tasks to computational resources will be done
at run-time. 

The data-driven architecture has been chosen because: (i) it potentially allows several avenues for implementing fault tolerance explicitly, for example: restarting processing based on data dependencies; data policies with regard to the loss of input or intermediate data; reallocation of work across the hardware. (ii) it does provide architectural separation between domain specific functionality, enclosing
it within the tasks, from the execution engine component which will exploit parallelism and will care about the performance,
efficiency and scalability of the parallel system. It is assumed that the tasks will be implemented for serial/multithreaded performance for a 
particular comuptational unit (gpu, cpu ...) (iii) a data-driven Execution Engine has the potential to provide scalability/performance without strong coupling
to the hardware architecture.

Decoupling EE from the architecture provides extensibility (possible performance portability) of the SDP software which will be required to accomodate upgrades. Enclosing the domain specific knowledge within the tasks will greatly improve maintainability, productivity and extensibility if a new algorithm development is required. To achieve scalability the EE itself needs to be scalable (i.e avoid bottlenecks at master node ...)

\subsubsection{Functional Architecture}
Figure 4 in the SDP Architecture document~\cite{SDParch} shows the Functional Architecture. It shows where the execution engine stands and it provides a complete picture of the functions that the SDP can perform. However, not all of this functionality will be associated with, or required for, a given scheduled observation. Therefore, for every observation (i.e. every 6 hours) a "driven program" needs to be generated
%taking into account the chosen capabilities\footnote{A SDP Processing Capability is the term chosen to represent the functionalities contained in the "driven program". It consists of connected pipelines (e.g. Receive, Imaging and Preserve) and
%supporting functions. See also the definition in the SDP Architecture document. Capabilities represent concrete instances of pipeline descriptions.}
. For this reason, we divide the Execution Engine for the SDP in two components:
\begin{enumerate}
\item The Generation of the driven program: This component is responsible for putting together the ?driven program? which will interface with the Core Execution Runtime and  be triggered by the Master Controler. The choice of Task granularity relies, therefore, in this component. Despite task granularity can also be reasoned a bit from within the Core Execution Runtime, (i.e. by analysing the task dependency graph some tasks may be merged), I think task granularity should be mainly reflected on the ?driven program?. 
\item The Core Execution Runtime:  The core execution runtime (runtime in short) will be responsible for performing data dependency analysis, schedule tasks when they are free of dependencies and manage data movement. Scheduling should aim at exploit data locality and load balance between computational units as well as minimize communication. These are the main functionalities of most available runtimes, however we expect this to be already a challenge due to the vast amount of data to be processed and therefore the high number of tasks that the runtime will need to manage. On top of these functionalities the runtime for SDP needs to: (i) Implement a fault tolerance mechanism which would allow to restart processing based on data dependencies; implement data policies with regard to the loss of input data or intermediate data (non-precious/precious data) and reallocate work across the hardware. 
%(ii) Generate QA (Quality Assesment) metrics in real time and on-demand. Processing of data needs to be monitored to avoid processing a wrong (i.e. corrupted) observation so QA metrics need to be generated in the form of "logging data" that will be processed by a function outside our program. This metrics will be generated by the tasks so the runtime has nothing to do other than taking into account that changing the genration of QA may impact on static load balancing.
\end{enumerate}

For the purpose of this document we focus on the Core Execution Runtime Component.



%We evaluate our tests in one of the following platforms:
%\begin{enumerate}
%\item Darwin and Wilkes\footnote{ http://www.hpc.cam.ac.uk}
%\item MareNostrum\footnote{https://www.bsc.es/marenostrum-support-services}
%\end{enumerate}


% split in two and organize better

\subsection{The Hardware}
% About the HW
The execution model runtime lyes in between the hardware architecture and the astronomical software.
About the hardware architecture we know/have assumed the following:

\begin{itemize}
\item Number of nodes to $N=12600$  [Old Cost Basis of Estimate]
\item HDD storage: 19PB shared between nodes
\item RAM storage: 806TB (64BG/node)
%\item The number of frequencies to $\Nfcorr=65000$ [SKA Baseline design v2]
\item Peak FLOPS capability of each node of 17.8 TeraFLOPS [Basis of Estimate]
\item Achieved FLOPS 25\% efficiency [Basis of Estimate]
\end{itemize}

\subsection{The astronomical software}
%About the pipelines
Regarding the SDP astronomical software, it has been defined as a set of pipelines, which are descrived in detail in here~\cite{SDPpipelines}.
In the following paragraphs I try to summarise the relevant aspects of this software from the execution engine point of view.
The amount of data to be processed depends on the specific pipeline and the telescope. 
SDP software will serve two telescopes SKA-Mid and SKA-Low (explain a bit more the purpose of this two !!). 

The data we are processing we will call visibility data. Visibility data consists of 

Data will be received from the telescope and placed in a buffer by a process called Ingress (or receive). The data is received in UDP using multiple 
40 GbE links. The incoming data rates depend on the telescope correlator's dump time (i.e. 0.14 s for SKA1-Low and 0.9 s for SKA1-Mid~\cite{ParametricModel}, we can assume a rate of 0.5Terabytes/s. After that, we can distinguish two different types of pipelines, which can be 
sumarized as follows:
\begin{itemize}
\item Fast imaging: Fast imaging pipeline needs to process the data in pseudo-real time as it is being received from the telescope.
 This means that every second we will have 0.5Terabytes of visibility data to process. The processing for this data is a map
\item Continumm imaging: Continum imaging pipeline reads data from the previous observation that has been stored in a buffer.
A double buffering technique is used, one buffer is being filled while the other is being processed. The size of the buffer is around 50PetaBytes
to hold two observations plus intermediate data products. It consists of an iterative process with two nested loops with all processing units needing
to synchronize at every iteration of the outer loop (with aproximately 10 iterations). The inner loop needs to converge (cleaning).
\item Calibration: Calibration pipeline runs alongside with continum pipeline, processing the same buffered data in a different way
\end{itemize}

% add more here about the computational demands and communication demands

On the other hand, we have analysed the pipelines in terms of data moment and computational requirements, using 
the parametric model (~\cite{ParametricModel} more information
can be found in \\ MISSING CITATION from Peter's work on the parametric model
and we learned the following:  \\ TODO





\section{Execution Engine Requirements}
\subsection{The data flow environment}

The structure of data: Introduce terminology and assumptions about how the program should work, a binary will be generated ...

\subsection{Requirements}

After analysing the SDP challenge we have defined the following list of requirements/challenges that will guide our comparison of existing execution engines:
\begin{itemize}
\item Data Layout. The programming model should provide a way to layout data that supports partitions for handling irregularly sampled collections.
Our visibility data requires more computation for higher frequency band so we would like to distribute data in a way that where high frequency band
visibilities are together the data chunk is smaller and for low frequencies the chunks are bigger to distribute computation evenly across the machine.
\item Memory management: Another challenge related to data occurs when the amount of data needed by a task exceeds the physical available memory. 
The programming model should provide a way to handle this situation (i.e. using files or mapping memory on demand ...) as transparent to the programmer as possible. Also related with memory management there will be the situation when the memory requirements of two tasks exceeds the amount of available physical memory, in that case runtime should take the memory demands into account during task scheduling so that tasks do not run together and prevent swapping.  
\item Scalability: we seek to achieve scalability up to 8K nodes.
\item Interoperability: Tasks may be written in  MPI, CUDA, Python?, OpenMP, OpenACC and Interface with convenient foreign functions  (i.e fft library)
\item I/O: a way to get data into the system (visibility data comes from the TM, streaming, and then from buffer, other data i.e. sky model, telescope state is in the storage service (file?)
\item performance: the overhead added by the runtime should be acceptable, performance comparable to an MPI program (ASKAPSoft). There is a timeout: the program should run before the next observation is available (6 hours). Factors that affect the overhead of a runtime are: communication overhead (we can measure latency and bandwith between two nodes and bi-sectional bandwidth using the full 8k nodes), runtime overhead (when communication is not an issue, we would like to measure performance of an embarassingly parallel program which should be comparable to SPMD program (i.e. MPI). Data transposition: A two-dimensional region will be used row-wise by a certain number of tasks but it will need to be used colum-wise by a different tasks, so, efectively it will require a transposition of the data, we would like to measure how the runtime can handle this (without explicitely using collective operations) 
\item resilience: the runtime must be stable enough to handle hardware failures, and take into account hints from the "driven program" which will tell the system the liveness of its k-inputs and output data (preciousness) and upon a failure of up to $m$ of the $k$ inputs compute a partial answer without an error condition.
\item Maintainability/ Extensibility: tasks and runtime should be separated by a well defined interface so that functionality can be extended with very little knowledge 
\item Performance Portability/ Extensibility: the runtime should be portable to different platforms and specifically should be able to run in a upgraded hardware (i.e with added memory per node, better network, more nodes ...) and take advantadge of the new available resources with minimal changes to the code.
\end{itemize}

\subsection{micro-benchmarks}

\subsection{Candidate systems}
The following systems are considered for this study:
\begin{enumerate}
\item Swift-T 
\item Regent\footnote{http://regent-lang.org/}/Legion\footnote{http://legion.stanford.edu/}
\item Parsec/Drague
\item COMPSS/OMPSS
\item Spark
\item StarPU
\item The Message Passing Interface(MPI)\footnote{http://www.mpi-forum.org} will also be studied,
(despite it is not a data-flow model) and used as a baseline in some cases.
\end{enumerate}


\section{Legion/Regent}

\subsection{Overview}

Legion is a data-centric programming system for writing portable high performance programs targeted at distributed heterogeneous architectures.
Legion presents abstractions which allow programmers to describe properties of program data (e.g. independence, locality). The Regent language provides a convenient way to program against the Legion runtime. Regent and Legion are active research projects at Stanford University.	

The legion programming model provides three important abstractions: 
\begin{enumerate}
\item Logical Regions: Logical regions are the fundamental abstraction used for describing program data in Legion applications. Logical regions are regarded as abstract data without considering its location in a particular memory system and without a fixed layout in memory. Each logical region is described by an index space of rows (or keys) (either unstructured pointers or structured 1D, 2D, or 3D arrays) and a field space of columns (or values). Regions can be recursively partitioned to match the hierarchical structure of memory and to facilitate parallel execution on subsets of data. Regions can also be partitioned multiple times to have different views of the data.  
%Regions support a relational model for data. 
% Data structures can be encoded in logical regions to express locality with partitioning and slicing describing data independence.
\item Tasks: A Regent program looks like sequential program with calls to tasks, which are functions that the programmer has marked as eligible for parallel execution. Regent guarantees sequential consistency. A program executes as a tree of tasks with a top-level task spawning sub-tasks which can recursively spawn further sub-tasks. All tasks in Legion must specify the logical regions they will access as well as the privileges (read, write, reduce) and coherence (only in Legion: exclusive, atomic, simultaneous or relaxed) for each logical region.
\item Mapping Interface: Legion makes no implicit decisions concerning how applications are mapped onto target hardware. Instead mapping decisions regarding how tasks are assigned to processors and how physical instances of logical regions are assigned to memories are made entirely by mappers. Mappers are part of application code and implement a mapping interface. Mappers are queried by the Legion runtime whenever any mapping decision needs to be made. Mappers are not exposed to Regent language.
\end{enumerate}

\subsection{Requirements overview}

Taking into account the list of requirements that have been identified for the SDP runtime we learned the following: 
\begin{itemize}
\item Data Layout. Visibility data should be irregularly distributed to balance computation. In Regent, Regions describe how data is used by the program. Regions can be arbitrarily partitioned into sub-regions based on index space or sliced on their field space. They can express partitions for handling irregularly sampled collections. Regions are not mapped to any physical memory position, this is done by the mapper which can only be expressed in Legion.
\item Memory management: Another challenge related to data occurs when the amount of data needed by a task exceeds the physical available memory. 
Regent allows the use of files (i.e in HDF5 format) that can be mapped to regions and used as such in the data dependency analysis, however the program should be aware that data would not fit in memory and use files instead. If programmer is not careful I do not know how would regent handle //TODO a test!! Also related with memory management there will be the situation when the memory requirements of two tasks exceeds the amount of available physical memory. In Regent tasks specify the regions they use and their permissions for those regions (weather the task performs reads, writes or reductions to each region). The information of how much data is needed by a specific task is known by the runtime and the scheduling decision happens at the mapper, however  mappers do not have view of the global scheduling state, i.e. I could not figure out how to see the available/free memory in a node so that this can be taken into account when scheduling tasks.   
\item Scalability: We seek to achieve scalability up to 8K cores. Legion has proven scalability up to 10K~\cite{MBauerPhD}, however Regent language scalability is limited to a small number of nodes (between 10 and 100)~\cite{Regent}. While Legion allows simultaneous access to regions (providing relaxed coherence ?). In regent each task has exclusive access to its region arguments, which limits scalability.
In Legion each task specifies for each region it accesses the privileges and coherence it requires of those regions. Privileges specify what the function can do with the regions while coherence specifies what other functions can do with the regions concurrently. Regent assumes all accesses to regions are exclusive which limits parallelism.
\item Interoperability: Regent provides interoperability with OpenMP,  MPI, CUDA, OpenACC and foreign functions  (i.e fft library). However when there is no way at the moment to request specific resources for a task to run (i.e number of cores, number of nodes...) which limits somehow the usefulness of this interoperability.
\item I/O: a way to get data into the system (visibility data comes from the TM, streaming, and then from buffer, other data i.e. sky model, telescope state is in the storage service (file?). Nothing prevents a task to read from a socket (I think ...), or a file (files can even be mapped to regions) to take into account data dependencies but that would not be necessary (talk to Danielle about how did they implement streaming in COMPSS)
\item performance: One factor that affect the overhead of a runtime is communication overhead. We have measured latency and bandwidth between two nodes and results are not very satisfying. Measurements in Darwin show that latency is around 5.5us. while the underlying communication system (GASNET MXM) achieves 2us latency. Regarding the bandwidth the theoretical peak in Darwin is around 7GB/s is fully utilized by GASNET MXM , from the Regent code we measured around 280MB/s. bi-sectional bandwidth using the full 8k nodes (TODO). runtime overhead (when communication is not an issue, we would like to measure performance of an embarassingly parallel program which should be comparable to SPMD program (i.e. MPI)(TODO). Data transposition: A two-dimensional region will be used row-wise by a certain number of tasks but it will need to be used colum-wise by a different tasks, so, efectively it will require a transposition of the data, we would like to measure how the runtime can handle this (without explicitely using collective operations). We need a better benchmark to mimic the problem because current benchmark implements the transpose of a matrix and legion optimizes the transposition of  data by allocating the input and output data blocks already transposed so that no data movement actually takes place only the transpose of a the blocks in local memory.
\item resilience: the runtime must be stable enough to handle hardware failures and take into account hints from the "driven program" which will tell the system the liveness of its k-inputs and output data (preciousness) and upon a failure of up to $m$ of the $k$ inputs compute a partial answer without an error condition. Legion runtime does not currently employ a distributed state machine mechanism, which makes fault tolerance difficult to implement. Right now, it is possible for a failed program to run until SLURM terminates it due to timeout.  
\item Maintainability/ Extensibility: tasks and runtime should be separated by a well defined interface so that functionality can be extended with very little knowledge. This is possible for Regent, plus it provides the mapper interface to add specific details about the scheduling, but it should only need to be written once and I do not know at this point if this is possible.
\item Performance Portability/ Extensibility: the runtime should be portable to different platforms and specifically should be able to run in a upgraded hardware (i.e with added memory per node, better network, more nodes ...) and take advantadge of the new available resources with minimal changes to the code. This should be ok too.
\end{itemize}

NOTE: Most of the information compiled in this section comes from the memo~\cite{RegentEval}, where more detailed analysis, justification as well as information on how to run the tests can be found.


%\section{Benchmarks}

%\section{Candidate systems evaluation}


\section{Conclusions}



\addcontentsline{toc}{section}{References}
%\bibliographystyle{egu}
\bibliography{ska}

\end{document}


