\section{OmpSs}

\subsection{Overview}

OmpSs supports dynamic offload of MPI tasks. MPI has the ability to implement an MPMD model by means of the MPI\_Comm\_spawn call. This API call enables
the dynamic spawn of new MPI processes on additional compute nodes that can run a different program, which is connected and can communicate with the original one.
The same functionality is implemented in OmpSs, it developes an API to dynamically allocate nodes/MPI processes, which returns a MPI intercommunicator that encloses
all the newly created MPI processes, OmpSs has been extended with a \emph{onto(comm,rank)} clause that specifies in which specific MPI process a task has to run, while
the compiler and runtime system manage all data transfer and synchronizations required.


\subsection{Requirements overview}

Taking into account the list of requirements that have been identified for the SDP runtime we learned the following: 
\begin{itemize}
\item Data Layout. 
\item Memory management: 
\item Interoperability: 
\item I/O: 
\item performance: 
\item resilience: 
\item Maintainability/ Extensibility: 
\item Performance Portability/ Extensibility: 
\end{itemize}

